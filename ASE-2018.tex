%% bare_conf.tex
%% V1.4b
%% 2015/08/26
%% by Michael Shell
%% See:
%% http://www.michaelshell.org/
%% for current contact information.
%%
%% This is a skeleton file demonstrating the use of IEEEtran.cls
%% (requires IEEEtran.cls version 1.8b or later) with an IEEE
%% conference paper.
%%
%% Support sites:
%% http://www.michaelshell.org/tex/ieeetran/
%% http://www.ctan.org/pkg/ieeetran
%% and
%% http://www.ieee.org/

%%*************************************************************************
%% Legal Notice:
%% This code is offered as-is without any warranty either expressed or
%% implied; without even the implied warranty of MERCHANTABILITY or
%% FITNESS FOR A PARTICULAR PURPOSE! 
%% User assumes all risk.
%% In no event shall the IEEE or any contributor to this code be liable for
%% any damages or losses, including, but not limited to, incidental,
%% consequential, or any other damages, resulting from the use or misuse
%% of any information contained here.
%%
%% All comments are the opinions of their respective authors and are not
%% necessarily endorsed by the IEEE.
%%
%% This work is distributed under the LaTeX Project Public License (LPPL)
%% ( http://www.latex-project.org/ ) version 1.3, and may be freely used,
%% distributed and modified. A copy of the LPPL, version 1.3, is included
%% in the base LaTeX documentation of all distributions of LaTeX released
%% 2003/12/01 or later.
%% Retain all contribution notices and credits.
%% ** Modified files should be clearly indicated as such, including  **
%% ** renaming them and changing author support contact information. **
%%*************************************************************************


% *** Authors should verify (and, if needed, correct) their LaTeX system  ***
% *** with the testflow diagnostic prior to trusting their LaTeX framework ***
% *** with production work. The IEEE's font choices and paper sizes can   ***
% *** trigger bugs that do not appear when using other class files.       ***                          ***
% The testflow support page is at:
% http://www.michaelshell.org/tex/testflow/


\label{beginning of document}
\documentclass[10pt,conference]{IEEEtran}
% Some Computer Society conferences also require the compsoc mode option,
% but others use the standard conference format.
%
% If IEEEtran.cls has not been installed into the LaTeX system files,
% manually specify the path to it like:
% \documentclass[conference]{../sty/IEEEtran}

\usepackage{graphicx}
	\graphicspath{{images/}} 
\renewcommand\IEEEkeywordsname{Keywords}
\usepackage{hyperref}
	\hypersetup{colorlinks=true,allcolors=blue}
\usepackage{hypcap}
\usepackage{subfig}
\usepackage{listings}
	\lstset{
  		basicstyle=\ttfamily\scriptsize,
  		frame=single,
  		breaklines=true,
  		numbers=left,
  		xleftmargin=2.5em,
  		framexleftmargin=0em,
  		emph={
        	class, extends, operation, abstract,
        	context, constraint, check,
        	for, if, return, true, and, ref,
        	message, in, package, val, attr, 
        	@link, @node, @compartment,
        	@namespace, @diagram
    	},
    	emphstyle=\textbf
	}
	\lstdefinestyle{interfaces}{
  		float=t
	}

% correct bad hyphenation here
\hyphenation{op-tical net-works semi-conduc-tor}


\begin{document}
%
% paper title
% Titles are generally capitalized except for words such as a, an, and, as,
% at, but, by, for, in, nor, of, on, or, the, to and up, which are usually
% not capitalized unless they are the first or last word of the title.
% Linebreaks \\ can be used within to get better formatting as desired.
% Do not put math or special symbols in the title.
\title{Hybrid Model Persistence}


% author names and affiliations
% use a multiple column layout for up to three different
% affiliations
\author{\IEEEauthorblockN{Alfa Yohannis\IEEEauthorrefmark{1}, Horacio Hoyos Rodriguez\IEEEauthorrefmark{2}, Fiona Polack\IEEEauthorrefmark{3}, Dimitris Kolovos\IEEEauthorrefmark{4}}
\IEEEauthorblockA{
    \IEEEauthorrefmark{1}\IEEEauthorrefmark{2}\IEEEauthorrefmark{4}Department of Computer Science, University of York, York, United Kingdom \\
    \IEEEauthorrefmark{3}School of Computing and Maths, Keele University, United Kingdom \\
Email: \IEEEauthorrefmark{1}ary506@york.ac.uk, \IEEEauthorrefmark{2}horacio\_hoyos\_rodriguez@ieee.org, \\
\IEEEauthorrefmark{3}f.a.c.polack@keele.a.cuk, \IEEEauthorrefmark{4}dimitris.kolovos@york.ac.uk}}

% conference papers do not typically use \thanks and this command
% is locked out in conference mode. If really needed, such as for
% the acknowledgment of grants, issue a \IEEEoverridecommandlockouts
% after \documentclass

% for over three affiliations, or if they all won't fit within the width
% of the page, use this alternative format:
% 
%\author{\IEEEauthorblockN{Michael Shell\IEEEauthorrefmark{1},
%Homer Simpson\IEEEauthorrefmark{2},
%James Kirk\IEEEauthorrefmark{3}, 
%Montgomery Scott\IEEEauthorrefmark{3} and
%Eldon Tyrell\IEEEauthorrefmark{4}}
%\IEEEauthorblockA{\IEEEauthorrefmark{1}School of Electrical and Computer Engineering\\
%Georgia Institute of Technology,
%Atlanta, Georgia 30332--0250\\ Email: see http://www.michaelshell.org/contact.html}
%\IEEEauthorblockA{\IEEEauthorrefmark{2}Twentieth Century Fox, Springfield, USA\\
%Email: homer@thesimpsons.com}
%\IEEEauthorblockA{\IEEEauthorrefmark{3}Starfleet Academy, San Francisco, California 96678-2391\\
%Telephone: (800) 555--1212, Fax: (888) 555--1212}
%\IEEEauthorblockA{\IEEEauthorrefmark{4}Tyrell Inc., 123 Replicant Street, Los Angeles, California 90210--4321}}




% use for special paper notices
%\IEEEspecialpapernotice{(Invited Paper)}




% make the title area
\maketitle

% As a general rule, do not put math, special symbols or citations
% in the abstract
\begin{abstract}
\label{abstract}
Lorem ipsum dolor sit amet consectetur adipiscing elit malesuada sollicitudin, penatibus ultricies primis accumsan volutpat id aliquet orci, netus congue tempus ligula proin ornare laoreet pharetra. Fringilla nibh felis odio duis tincidunt eget ultricies tellus, eros molestie phasellus lacinia accumsan mus gravida conubia, purus at posuere tristique porttitor volutpat sociis. Varius duis facilisi condimentum rhoncus nascetur velit cum nostra, cubilia ridiculus a himenaeos massa sem inceptos, dignissim in nibh elementum interdum nisi ligula.
\end{abstract}
%
% no keywords
\begin{IEEEkeywords} 
model persistence, hybrid persistence, change-based persistence, state-based persistence
\end{IEEEkeywords}

% For peer review papers, you can put extra information on the cover
% page as needed:
% \ifCLASSOPTIONpeerreview
% \begin{center} \bfseries EDICS Category: 3-BBND \end{center}
% \fi
%
% For peerreview papers, this IEEEtran command inserts a page break and
% creates the second title. It will be ignored for other modes.
\IEEEpeerreviewmaketitle

\section{Introduction}
\label{sec:introduction}
Change-based model persistence \cite{DBLP:conf/models/YohannisKP17} comes with at least two main advantages: it provides support for (1) fast comparison of versions of the same model \cite{DBLP:conf/sde/LippeO92,DBLP:conf/caise/IgnatN05,DBLP:conf/edoc/KoegelHLHD10,koegel2010emfstore}  -- which can also substantially speed up incremental model management activities, and (2) recording fine-grained changes in models that can enable model analytics (e.g. understand how modellers use modelling languages and tools) \cite{DBLP:journals/entcs/RobbesL07}. However, the approach comes with the cost that we need to pay: ever-growing model files \cite{DBLP:conf/edoc/KoegelHLHD10,DBLP:journals/entcs/RobbesL07} and increased loading time \cite{mens2002state}. In this work we are addressing the former by introducing the concept of hybrid model persistence. In hybrid model persistence the change-based representation is augmented with a state-based representation of the latest snapshot of the model which is used to speed up model loading and querying. In this work, we augment our change-based model persistence with NeoEMF \cite{daniel2016neoemf} as the state-based representation. 

\section{Hybrid Persistence}
\label{sec:hybrid_persistence}
Explain the current CBP/NeoEMF implementation and how only the latter is used for querying but changes are recorded to both.




\section{Evaluation}
\label{sec:evaluation}
- For evaluation, show the loading/saving/space overhead of CBP/NeoEMF
vs. pure NeoEMF as well as the performance benefits in terms of model
comparison and argue that the former is a fair trade-off for the
latter (I suspect that state-based comparison will explode
sufficiently large models)

\subsection{Model Loading Time}
\label{sec:model_loading_time}

\subsection{Model Saving Time}
\label{sec:model_saving_time}

\subsection{Memory Footprint for Loading Model}
\label{sec:memory_footprint_for_loading_model}

\subsection{Memory Footprint for Saving Model}
\label{sec:memory_footprint_for_saving_model}

\subsection{Storage Space Usage}
\label{sec:storage_space_usage}

\section{Discussion}
\label{sec:discussion}


\section{Literature Review}
\label{sec:literature_review}
 There are several non-XMI approaches to state-based model persistence, using relational or NoSQL databases. For example, EMF Teneo\,\cite{eclipse2017teneo} persists EMF models in relational databases, while Morsa \cite{DBLP:conf/models/Espinazo-PaganCM11} and NeoEMF \cite{daniel2016neoemf} persist models in document and graph databases, respectively.  None of these approaches provides built-in support for versioning and models are eventually stored in binary files/folders which are known to be a poor fit for text-oriented version control systems like Git and SVN. Connected Data Objects (CDO) \cite{eclipse2017cdo}, provides support for database-backed model persistence as well as collaboration facilities, but its adoption necessitates the use of a separate version control system in the software development process (e.g. a Git repository for code and a CDO repository for models), which introduces fragmentation and administration challenges \cite{barmpis2014evaluation}. Similar challenges arise in relation to other model-specific version control systems such as EMFStore\,\cite{koegel2010emfstore}.

\section{Conclusions and Future Work}
\label{sec:conlcusions_and_future_work}


% An example of a floating figure using the graphicx package.
% Note that \label must occur AFTER (or within) \caption.
% For figures, \caption should occur after the \includegraphics.
% Note that IEEEtran v1.7 and later has special internal code that
% is designed to preserve the operation of \label within \caption
% even when the captionsoff option is in effect. However, because
% of issues like this, it may be the safest practice to put all your
% \label just after \caption rather than within \caption{}.
%
% Reminder: the "draftcls" or "draftclsnofoot", not "draft", class
% option should be used if it is desired that the figures are to be
% displayed while in draft mode.
%
%\begin{figure}[!t]
%\centering
%\includegraphics[width=2.5in]{myfigure}
% where an .eps filename suffix will be assumed under latex, 
% and a .pdf suffix will be assumed for pdflatex; or what has been declared
% via \DeclareGraphicsExtensions.
%\caption{Simulation results for the network.}
%\label{fig_sim}
%\end{figure}

% Note that the IEEE typically puts floats only at the top, even when this
% results in a large percentage of a column being occupied by floats.


% An example of a double column floating figure using two subfigures.
% (The subfig.sty package must be loaded for this to work.)
% The subfigure \label commands are set within each subfloat command,
% and the \label for the overall figure must come after \caption.
% \hfil is used as a separator to get equal spacing.
% Watch out that the combined width of all the subfigures on a 
% line do not exceed the text width or a line break will occur.
%
%\begin{figure\\}[!t]
%\centering
%\subfloat[Case I]{\includegraphics[width=2.5in]{box}%
%\label{fig_first_case}}
%\hfil
%\subfloat[Case II]{\includegraphics[width=2.5in]{box}%
%\label{fig_second_case}}
%\caption{Simulation results for the network.}
%\label{fig_sim}
%\end{figure\\}
%
% Note that often IEEE papers with subfigures do not employ subfigure
% captions (using the optional argument to \subfloat[]), but instead will
% reference/describe all of them (a), (b), etc., within the main caption.
% Be aware that for subfig.sty to generate the (a), (b), etc., subfigure
% labels, the optional argument to \subfloat must be present. If a
% subcaption is not desired, just leave its contents blank,
% e.g., \subfloat[].


% An example of a floating table. Note that, for IEEE style tables, the
% \caption command should come BEFORE the table and, given that table
% captions serve much like titles, are usually capitalized except for words
% such as a, an, and, as, at, but, by, for, in, nor, of, on, or, the, to
% and up, which are usually not capitalized unless they are the first or
% last word of the caption. Table text will default to \footnotesize as
% the IEEE normally uses this smaller font for tables.
% The \label must come after \caption as always.
%
%\begin{table}[!t]
%% increase table row spacing, adjust to taste
%\renewcommand{\arraystretch}{1.3}
% if using array.sty, it might be a good idea to tweak the value of
% \extrarowheight as needed to properly center the text within the cells
%\caption{An Example of a Table}
%\label{table_example}
%\centering
%% Some packages, such as MDW tools, offer better commands for making tables
%% than the plain LaTeX2e tabular which is used here.
%\begin{tabular}{|c||c|}
%\hline
%One & Two\\
%\hline
%Three & Four\\
%\hline
%\end{tabular}
%\end{table}


% Note that the IEEE does not put floats in the very first column
% - or typically anywhere on the first page for that matter. Also,
% in-text middle ("here") positioning is typically not used, but it
% is allowed and encouraged for Computer Society conferences (but
% not Computer Society journals). Most IEEE journals/conferences use
% top floats exclusively. 
% Note that, LaTeX2e, unlike IEEE journals/conferences, places
% footnotes above bottom floats. This can be corrected via the
% \fnbelowfloat command of the stfloats package.



% conference papers do not normally have an appendix


% use section\\ for acknowledgment
\section*{Acknowledgments}
This research is part of a doctoral programme funded by \emph{Lembaga Pengelola Dana Pendidikan Indonesia} (Indonesia Endowment Fund for Education).

% trigger a \newpage just before the given reference
% number - used to balance the columns on the last page
% adjust value as needed - may need to be readjusted if
% the document is modified later
%\IEEEtriggeratref{8}
% The "triggered" command can be changed if desired:
%\IEEEtriggercmd{\enlargethispage{-5in}}

% references section

% can use a bibliography generated by BibTeX as a .bbl file
% BibTeX documentation can be easily obtained at:
% http://mirror.ctan.org/biblio/bibtex/contrib/doc/
% The IEEEtran BibTeX style support page is at:
% http://www.michaelshell.org/tex/ieeetran/bibtex/
%\bibliographystyle{IEEEtran}
% argument is your BibTeX string definitions and bibliography database(s)
%\bibliography{IEEEabrv,../bib/paper}
%
% <OR> manually copy in the resultant .bbl file
% set second argument of \begin to the number of references
% (used to reserve space for the reference number labels box)
%\begin{thebibliography}{1}
%
%\bibitem{IEEEhowto:kopka}
%H.~Kopka and P.~W. Daly, \emph{A Guide to \LaTeX}, 3rd~ed.\hskip 1em plus
%  0.5em minus 0.4em\relax Harlow, England: Addison-Wesley, 1999.
%
%\end{thebibliography}

\bibliographystyle{IEEEtran}
\bibliography{references}



% that's all folks
\end{document}